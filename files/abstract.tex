%% Copyright 1998 Pepe Kubon
%%
%% `abstract.tex' --- abstract for thes-full.tex, thes-short-tex from
%%                    the `csthesis' bundle
%%
%% You are allowed to distribute this file together with all files
%% mentioned in READ.ME.
%%
%% You are not allowed to modify its contents.
%%

%%%%%%%%%%%%%%%%%%%%%%%%%%%%%%%%%%%%%%%%%%%%%%%%%
%
%       Abstract
%
%%%%%%%%%%%%%%%%%%%%%%%%%%%%%%%%%%%%%%%%%%%%%%%%

\prefacesection{Abstract}

LLVM as a compiler tool chain is gaining popularity recently and it is a good tool for parallel bit streams (Parabix), because first it provides a machine-independent intermedia representation (IR) as its virtual instruction set so the code portability can be easily achieved; second it supports interprocedural optimizations that are aware of both the target and the context of certain Parabix operation that enables better machine code generation; third it provides just-in-time compiling facility to support run-time source generation which is important for regular expression matching. However, the native backend of LLVM is not ideal and it generates slow machine code for Parabix. To address this problem, we extend LLVM in this thesis to systematically support parallel bit streams. We first redefine the type legality in LLVM and implement a few algorithms to lower vectors with small elements; we then insert logic in LLVM backend to recognize and properly handle Parabix critical operations such as packing, merging and long stream addition. Our experiment on the X86 architecture demonstrates the performance as good as the well-tuned IDISA library and about 300 times faster than the LLVM native backend for some micro benchmark. We also demonstrate a better performance gained by switching from X86 SSE2 to AVX2 without any change in the source code.














