%% Copyright 1998 Pepe Kubon
%%
%% `abstract.tex' --- abstract for thes-full.tex, thes-short-tex from
%%                    the `csthesis' bundle
%%
%% You are allowed to distribute this file together with all files
%% mentioned in READ.ME.
%%
%% You are not allowed to modify its contents.
%%

%%%%%%%%%%%%%%%%%%%%%%%%%%%%%%%%%%%%%%%%%%%%%%%%%
%
%       Abstract
%
%%%%%%%%%%%%%%%%%%%%%%%%%%%%%%%%%%%%%%%%%%%%%%%%

\prefacesection{Abstract}

Parabix (parallel bit stream) technology uses the SIMD (single-instruction
multiple-data) capabilities of commodity processors for high-performance
text processing in applications such as Unicode transcoding, XML parsing
and regular expression matching.   LLVM is a widely-used compiler-infrastructure
with a target-independent intermediate representation (IR) that includes
notational support for SIMD operations on vectors of small integers.
This thesis investigates whether it is possible to modify LLVM to
incorporate all the SIMD processing requirements of Parabix both to
increase the portability of applications and to create
additional opportunities for optimization of those operations in the
context of code generation.   Our modifications to LLVM include
redefining type legality and lowering for vectors of small elements
as well as insertion of logic to recognize and properly handle Parabix-critical
operations such as packing, merging and long stream addition.
Experiments on the X86/SSE2 architecture show a speedup over the unmodified
LLVM of about 300 times for some micro-benchmarks and demonstrate
Parabix application performance slightly better than with its original
SSE2 libraries. We also demonstrate performance scaling in switching
from X86/SSE2 to X86/AVX2 without any change in source code.
