%%%%%%%%%%%%%%%%%%%%%%%%%%%%%%%%%%%%%%%%%%%%%%%%%
%
%     Chapter 7
%
%%%%%%%%%%%%%%%%%%%%%%%%%%%%%%%%%%%%%%%%%%%%%%%%

\chapter{Conclusion}
\label{seven}

In this thesis, we demonstrated that it is possible to extend LLVM type system to support Parabix technology. We have shown systematic support of the vector of $i2^k$ and support of critical Parabix operations in the target-independent IR library. We have also shown in one specific target: Intel X86, we can generate efficient native code. We added a new LLVM intrinsic that enables chained additions on long bit streams, which can be used for a broad category of applications.

With the LLVM backend, Parabix technology has better chances to target at different platforms efficiently such as PowerPC servers and the ARM mobile platform. Further extension of the LLVM code generation may be needed but it can all be contributed the LLVM community and benefit a variety of applications. (More future work maybe).

